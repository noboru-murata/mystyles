\documentclass[
  leqno,
  twoside,
  numbers=noenddot,
  % headsepline, % koma-script の機能でヘッダ線を付ける
  % footsepline, % koma-script の機能でフッタ線を付ける
]{scrbook}
\usepackage[style=plain]{mybook}
\myPageStyle{lines}

%% 以下でもほぼ等価 ()章のタイトル頁の扱いを除く)
% \documentclass[
%   leqno,
%   twoside,
%   numbers=noenddot,
%   headsepline, % koma-script の機能でヘッダ線を付ける
%   footsepline, % koma-script の機能でフッタ線を付ける
% ]{scrbook}
% \usepackage[style=plain]{mybook}
% \myPageStyle{scrheadings}

%% 章の表記を変えたい場合はここから
\makeatletter 
\renewcommand{\chapterlinesformat}[3]{%
  \renewcommand*{\thechapter}{\chaptername{\@arabic\c@chapter\kern.1em}}
  % \renewcommand*{\thechapter}{{第}\@arabic\c@chapter{章}}
  \@hangfrom{#2}{#3}%
}
\makeatother
%% ここまでを使う

\usepackage[%
  backend=biber,
  bibencoding=latin1,
  style=ieee, % ieee, nature, numeric, authoryear いろいろある
  url=false, % 余計な項目は表示しない
  isbn=false,
  doi=false,
  eprint=false,
]{biblatex}
\AtEveryBibitem{\clearfield{note}} % note項目を表示しない
\addbibresource{books.bib} 
% \addbibresource{papers.bib} % databaseを追加する場合

%%% 必要なpackageの読み込みの例
\usepackage{graphicx}
\graphicspath{{./imgs/}}
\usepackage{bxjalipsum} % ダミーの文書
\usepackage[math]{blindtext} % ダミーの文書
%%% packageの例の終わり

%%% タイトル
\subject{研究報告書}
\title{「我輩」の秘密に関する研究}
\subtitle{私が彼について知っている2,3の事柄} 
\author{夏目 漱石}
\date{\today}
\publishers{
  早稲田大学 先進理工学研究科\\
  電気・情報生命専攻
}

%%% 本文
\begin{document}
\frontmatter
\maketitle
\tableofcontents

\mainmatter
%%% 以下本文の例
\chapter{はじめに}

\section{まえおき}
\jalipsum{preamble}
\footnote{\jalipsum{hatsukoi}}

\section{過去の事例}
\blindmathpaper 
%% 数式がどうなるか出してみる

\chapter{本文}
%% 実際に各章は include すればよい
%% \include{1_Introduction} 

\section{我輩のこと}
\jalipsum[1-3]{wagahai}
\footnote{\jalipsum{iroha}}

\jalipsum[4-6]{wagahai}
\footnote{例えば\cite{吉田2006,竹内1963,杉浦1980,杉浦1985,田中2006}を参照}

\section{主人の不平}
\jalipsum[7-12]{wagahai}
%% 図を入れてみる
\GraphFile{sample_figs} % 紙芝居をする場合
\begin{figure}[htbp] % 普通の環境
  \centering
  \myGraph[1]{} % linewidthの何倍か指定
  \caption{東京都の陽性患者数の推移.
    緑は7日移動平均,橙は14日移動平均を表す.
    \label{fig:1}}
\end{figure}

\section{車屋の黒猫}
\jalipsum[13-18]{wagahai}

\begin{figure} % 2つ並べてみる
  \centering
  \myGraph{状態空間モデルによる各成分の推定}
  \myGraph{状態空間モデルによる平均の推定}
  \caption{状態空間モデルの推定.
    \label{fig:2}}
\end{figure}

\section{鼠以外の御馳走}
\jalipsum[19-24]{wagahai}
\begin{figure} % マージンには置けないのでそのまま配置
  \centering
  \myGraph*{} % 幅に合わせて表示
  \caption{$Q_{\mathrm{cycle}}$ の検討について.
    \label{fig:4}}
\end{figure}

\section{放蕩家について}
\jalipsum[25-]{wagahai}

\chapter{まとめ}

\section{これまで}
\jalipsum[1-5]{kusamakura}
\footnote{\jalipsum{jugemu}}

\jalipsum[6-7]{kusamakura}

\section{これから}
\jalipsum[8-]{kusamakura}
\cite{Wasserman2004,Billingsley1999,HornJohnson1990,Hastie_etal2009}

\backmatter
\begin{otherlanguage}{english}
  % babel系が若干悪さをするので英語にして回避
  \printbibliography[title=参考文献]
\end{otherlanguage}

\end{document}
